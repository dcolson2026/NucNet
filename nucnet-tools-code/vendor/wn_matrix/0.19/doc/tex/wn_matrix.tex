%///////////////////////////////////////////////////////////////////////////////
% <file
%   repository_path  = "$Source: /pub/cvsprojects/nucleo/web/home/modules/wn_matrix/doc/tex/wn_matrix.tex,v $"
%   revision         = "$Revision: 1.14 $"
%   date             = "$Date: 2008/06/04 15:28:10 $"
%   tag              = "$Name: wn_matrix-0-24 $"
%   template_version = "cp_template.0.12"
% >
%
% <license>
%   See the README_module.xml file for this module for copyright and license
%   information.
% </license>

%   <description>
%     <abstract>
%       This is the Webnucleo Report for the wn_matrix Module.
%     </abstract>
%     <keywords>
%       wn_matrix Module, webnucleo report
%     </keywords>
%   </description>
%
%   <authors>
%     <current>
%       <author userid="dcadams" start_date="2006/05/11" />
%     </current>
%     <previous>
%     </previous>
%   </authors>
%
%   <compatibility>
%     TeX (Web2C 7.4.5) 3.14159 kpathsea version 3.4.5
%   </compatibility>
%
% </file>
%///////////////////////////////////////////////////////////////////////////////
%
% This is a sample LaTeX input file.  (Version of 9 April 1986)
%
% A '%' character causes TeX to ignore all remaining text on the line,
% and is used for comments like this one.

\documentclass{article}    % Specifies the document style.

\usepackage{hyperref}

                           % The preamble begins here.
\title{Webnucleo Technical Report: wn\_matrix Module}  

\author{David Adams and Bradley S. Meyer}
%\date{December 12, 1984}   % Deleting this command produces today's date.

\begin{document}           % End of preamble and beginning of text.

\maketitle                 % Produces the title.

The following will provide information regarding the routines that comprise
the wn\_matrix Module. 
The wn\_matrix Module is divided into two parts: routines that deal with 
creating and managing the WnMatrix structure, and routines for performing matrix
operations with the WnMatrix structures. 
This report describes the WnMatrix structures and
details of some of the wn\_matrix routines.
The module itself can be found at:
\begin{center}
http://www.webnucleo.org/home/modules/wn\_matrix/
\end{center}

\section{wn\_matrix Structures Overview}  
                                       
{\em WnMatrix} is a structure for storing a matrix.  As of version 0.2,
elements of the matrix are stored in a multi-dimensional hash,
which provides excellent flexibility in
adding and removing elements from the matrix.
The hash routines used are those from libxml, the xml C parser and toolkit of
the Gnome.  Version 0.1, now no longer supported, used doubly-linked lists,
which could require $\cal O$(N) operations,
where N is the number of columns in the
matrix, to store and retrieve elements.  The hash only requires $\cal O$(1)
operations.

The WnMatrix structure is central to the module, as the majority of
the routines contained 
within the module are used to either operate on the structure or retrieve 
information about its contents.  The structure itself is principally comprised
of a pointer to a libxml xmlHashTable pointer, in which the non-zero elements
are stored.  Other structure elements are unsigned ints containing the
number of rows and columns in the matrix (the number of non-zero elements
is not kept but rather is retrieved by the libxml xmlHashSize routine).
Finally, there is a pointer to an internal data structure for use in callback
routines on the hash.

As of version 0.3, the clear and scale matrix routines are correct.
Version 0.2 used hash callbacks for these routines but modified the hash
during the callbacks.  This naturally led to undefined behavior.  Version
0.3 also adds getCopy and getTranspose routines to the API.  These return
new matrices that the caller must free with WnMatrix\_\_free when no longer
needed.


{\em WnMatrix\_\_Line} is a structure for storing data relevant to the non-zero
elements of a line in the matrix, that is, a row or a column.  These data
are stored in the structure as arrays.

As of release 0.4, we have introduced three new structures, namely,
{\em WnMatrix\_\_Coo}, {\em WnMatrix\_\_Csr}, and {\em WnMatrix\_\_Yale}
for storing the sparse matrix in coordinate, compressed sparse row, and Yale
sparse matrix format, respectively.  We have also rearranged the API to
accomodate these changes, which means there is a backward incompatibilty between
release 0.4 and earlier versions.  The reason for the change is to relieve the
user of the burden of allocating memory for these alternative sparse matrix
formats.  Thus, for example, when the user calls {\em WnMatrix\_\_getCoo()},
a pointer to a coordinate matrix is returned.  The {\em WnMatrix\_\_getCoo()}
routine does all the memory allocation.  The routine
{\em WnMatrix\_\_Coo\_\_getRowVector()} then returns the coordinate row matrix
array (with a number of elements equal to the number of non-zero elements
in the matrix).  The user frees the memory for the coordinate matrix (and,
consequently, the row array) by calling the {\em WnMatrix\_\_Coo\_\_free()}
routine.  Example codes in the src/examples directory of the distribution
demonstrate how this works.

Our philosophy has been that the user should not have to worry about the
data structures used by wn\_matrix.  Instead, we intend that the user
should interact with the wn\_matrix structures via the API routines.
For this reason, the API does not make the content of
the wn\_matrix structures public.
Nevertheless, the interested user may find their prototypes located in the 
WnMatrix header file {\tt WnMatrix.h}.

\section{wn\_matrix Routines}

For documentation on the wn\_matrix routines, see the WnMatrix.h file
in the Overview in the Technical Resources for the current release.
The documentation provides a brief description of each routine, the prototype,
pre- and post-conditions, and examples on using the routines.

As of release 0.5, we have introduced routines to read input matrix data
from XML and output matrix data to XML.  For example, the routine
{\em WnMatrix\_\_new\_from\_xml()} inputs matrix data in row, column, value
triplets from an XML file and creates a matrix based on those data.  The
routines {\em WnMatrix\_\_Coo\_\_writeToXmlFile()},
{\em WnMatrix\_\_Coo\_\_writeToXmlFile()}, and
{\em WnMatrix\_\_Coo\_\_writeToXmlFile()} output coordinate, compressed sparse
row, and Yale matrix forms of the matrix to XML output.  A new routine
also allows the user to validate the input XML file
against Webnucleo.org's input matrix
\href{http://www.webnucleo.org/home/modules/wn_matrix/0.5/xsd_pub/coordinate_matrix.xsd}{schema}
for version 0.5 (future releases will have similar xsd\_pub directories).
For more details, the user should consult the API documentation.

As of version 0.7, wn\_matrix uses the gnu gsl scientific library vector
structure when interacting with vectors.  This change of course means that
wn\_matrix now depends on gsl as well as libxml.  We have added routines to
parse in vector data from an xml file
({\em WnMatrix\_\_new\_gsl\_vector\_from\_xml()}),
to validate the xml ({\em WnMatrix\_\_is\_valid\_vector\_input\_xml()}),
and to dump a gsl\_vector to an xml file
({\em WnMatrix\_\_write\_gsl\_vector\_to\_xml\_file()}).
While it is possible to interact with the elements of the gsl\_vector data
structure directly (see the gsl documentation),
we have also added API routines to get the size of the
vector ({\em WnMatrix\_\_get\_gsl\_vector\_size()}) and to retrieve the data
array from the gsl\_vector ({\em WnMatrix\_\_get\_gsl\_vector\_array()}).

Our change over to using gsl\_vectors simplifies the wn\_matrix API for
the matrix times vector and transpose matrix times vector routines.
It nevertheless introduces a backwards incompatibility.  While we could have
kept the routines\\
\\{\em WnMatrix\_\_getMatrixTimesVector()}\\
\\and\\
\\{\em WnMatrix\_\_getTransposeMatrixTimesVector()}\\ \\
and deprecated them, we
decided to remove them and replace them with\\
\\{\em WnMatrix\_\_computeMatrixTimesVector()}\\ \\
and\\ \\
{\em WnMatrix\_\_computeTransposeMatrixTimesVector()}.\\ \\
This is desirable since the old routines did not in fact retrieve (get)
pre-existing data but rather computed them.  The new routines not only
have simplified calls because they use gsl\_vectors,
but their names also better reflect their functionality.

Because wn\_matrix now depends on gsl, it made sense in version 0.7
for us also to replace
{\em WnMatrix\_\_getDenseMatrix()} that returned a double** with
{\em WnMatrix\_\_getGslMatrix()} that returns a gsl\_matrix *.  This makes
for a cleaner interface in the API and easier allocations.  Also, we
added {\em WnMatrix\_\_solve()}, a routine that solves a matrix equation
$A x = b$, given input vector $A$ and right-hand-side vector $b$.  The
routine uses gsl linear algebra routines to solve the equation.

\end{document}
