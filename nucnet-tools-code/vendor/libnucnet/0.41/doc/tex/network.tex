%///////////////////////////////////////////////////////////////////////////////
% <file type="public">
%
% <license>
%   See the README_module.xml file for this module for copyright and license
%   information.
% </license>
%   <description>
%     <abstract>
%       This is the Webnucleo Report for the libnucnet Module.
%     </abstract>
%     <keywords>
%       libnucnet Module, webnucleo report
%     </keywords>
%   </description>
%
%   <authors>
%     <current>
%       <author userid="mbradle" start_date="2007/08/28" />
%     </current>
%     <previous>
%     </previous>
%   </authors>
%
%   <compatibility>
%     TeX (Web2C 7.4.5) 3.14159 kpathsea version 3.4.5
%   </compatibility>
%
% </file>
%///////////////////////////////////////////////////////////////////////////////
%
% This is a sample LaTeX input file.  (Version of 9 April 1986)
%
% A '%' character causes TeX to ignore all remaining text on the line,
% and is used for comments like this one.

\documentclass{article}    % Specifies the document style.

\usepackage[dvips]{graphicx}
\usepackage{hyperref}

                           % The preamble begins here.
\title{Webnucleo Technical Report: Network Calculations with libnucnet}  % Declares the document's title.

\author{Bradley S. Meyer}
%\date{December 12, 1984}   % Deleting this command produces today's date.

\begin{document}           % End of preamble and beginning of text.

\maketitle                 % Produces the title.


This technical report describes some details of solving nuclear
reaction networks with the libnucnet module.

\section{The Network Equations}

A nuclear reaction network is a collection of nuclear species and
reactions among them.  We denote the number density of species $i$
in the network by $n_i$.  It is convenient to define the number
density as
\begin{equation}
n_i = \rho N_A Y_i,  \label{eq:abund}
\end{equation}
where $\rho$ is the mass density (grams/cm$^3$), $N_A$ is Avogadro's
number, and $Y_i$ is the abundance of species $i$ per nucleon.

The number density of a species changes with time both due to
nuclear reactions and to a change in the volume of the material. For
purposes of illustration, let us assume a simple three-element
network with species 1, 2, and 3 and two reactions:
\begin{equation}
1 + 2 \rightleftharpoons 3 + \gamma \label{eq:reaction1}
\end{equation}
and
\begin{equation}
2 \to 1  \label{eq:reaction2}
\end{equation}
The former reaction is a radiative capture reaction (with its
reverse disintegration).  The latter reaction is a decay reaction.

The differential equation describing the rate of change of species 1
is:
\begin{equation}
\frac{dn_1}{dt} = -n_1 n_2 \langle \sigma v \rangle + \lambda_\gamma
n_3 + \lambda n_2 - \frac{n_1}{\tau}.  \label{eq:diffeq1}
\end{equation}
Here $\sigma$ is the interaction cross section for a pair of nuclei
of type 1 and 2.  $v$ is the relative velocity of the interacting
nuclei.  The term $\langle \sigma v \rangle$ is thus the thermally
averaged cross section for the interaction of a pair of nuclei of
type 1 and 2.  By multiplying this term by the number densities of
the nuclei of type 1 and 2, we get the total number of interactions
of nuclei of type 1 and 2 per unit volume per unit time.   The
negative sign indicates that such interactions decrease the number
density of species 1.  The term $\lambda_\gamma$ is the rate of
disintegration of a nucleus of type 3.  This is the reverse reaction
to the first and increases the number of nuclei of type 1 (and 2).
The term $\lambda$ is the rate for a nucleus of type 2 to decay into
a nucleus of type 1.  Finally, the term $-\frac{n_1}{\tau}$
represents the decrease in the number density of species $i$ simply
due to the expansion of the system.  The expansion timescale $\tau$
is defined by
\begin{equation}
\frac{1}{\tau} \equiv -\frac{1}{\rho}\frac{d\rho}{dt}.
\label{eq:tau}
\end{equation}

The differential equations for the number densities of species 2 and
3 are:
\begin{equation}
\frac{dn_2}{dt} = -n_1 n_2 \langle \sigma v \rangle + \lambda_\gamma
n_3 - \lambda n_2 - \frac{n_2}{\tau}  \label{eq:diffeq2}
\end{equation}
and
\begin{equation}
\frac{dn_3}{dt} = n_1 n_2 \langle \sigma v \rangle - \lambda_\gamma
n_3 - \frac{n_3}{\tau}. \label{eq:diffeq3}
\end{equation}

It is convenient to remove the dependence on density by substituting
Eqs. (\ref{eq:abund}) and (\ref{eq:tau}) in Eqs. (\ref{eq:diffeq1}),
(\ref{eq:diffeq2}), and (\ref{eq:diffeq3}).  The results are:
\begin{equation}
\frac{dY_1}{dt} = -N_A \langle \sigma v \rangle \rho Y_1 Y_2 +
\lambda_\gamma Y_3 + \lambda Y_2 \label{eq:diffYeq1}
\end{equation}
and
\begin{equation}
\frac{dY_2}{dt} = -N_A \langle \sigma v \rangle \rho Y_1 Y_2 +
\lambda_\gamma Y_3 - \lambda Y_2 \label{eq:diffYeq2}
\end{equation}
and
\begin{equation}
\frac{dY_3}{dt} = N_A \langle \sigma v \rangle \rho Y_1 Y_2 -
\lambda_\gamma Y_3. \label{eq:diffYeq3}
\end{equation}
These are the appropriate forms of the network equations to solve.  We note 
that the equations must be modified to account for identical reactants
or products.  libnucnet routines automatically account for these.

\section{The Matrix}

If we know the abundances of the species at time $t$, we seek their
abundances at time $t + \Delta t$.  This involves integrating the
coupled differential equations of the type in Eqs.
(\ref{eq:diffYeq1}-\ref{eq:diffYeq3}) over time step $\Delta t$.
Because the system of equations is very stiff (i.e., the shortest
timescale is much less than the longest), we need to solve the
equations implicitly.  To do so, we first finite difference in time
and use the definition
\begin{equation}
\Delta Y_i = Y_i (t + \Delta t) - Y_i (t).  \label{eq:deltaY}
\end{equation}
Implicit differentiation means that we compute derivatives at time
$t + \Delta t$.  In this case, Eq. (\ref{eq:diffYeq1}) becomes
\begin{equation}
\frac{\Delta Y_1}{\Delta t} = -N_A \langle \sigma v \rangle \rho
Y_1 ( t + \Delta t ) Y_2 ( t + \Delta t ) + \lambda_\gamma Y_3 ( t + \Delta t )
+ \lambda  Y_2 ( t + \Delta t )
\end{equation}
where the rates (such as $N_A \langle
\sigma v \rangle \rho$) are computed at time $t + \Delta t$.  We may then
rearrange this equation to read
\[
f_1 = \left( Y_1( t + \Delta t) - Y_1( t ) \right )/
{\Delta t} + N_A \langle \sigma v \rangle \rho
Y_1 ( t + \Delta t) Y_2 ( t + \Delta t )
\]
\begin{equation}
-\lambda_\gamma Y_3 ( t + \Delta t ) - \lambda Y_2 ( t + \Delta t ) = 0.
\label{eq:f1}
\end{equation}
Similarly, we find
\[
f_2 = \left( Y_2( t + \Delta t) - Y_2( t ) \right )/
{\Delta t} + N_A \langle \sigma v \rangle \rho
Y_1 ( t + \Delta t) Y_2 ( t + \Delta t )
\]
\begin{equation}
-\lambda_\gamma Y_3 ( t + \Delta t ) + \lambda Y_2 ( t + \Delta t ) = 0.
\label{eq:f2}
\end{equation}
and
\[
f_3 = \left( Y_3( t + \Delta t) - Y_3( t ) \right )/
{\Delta t} - N_A \langle \sigma v \rangle \rho
Y_1 ( t + \Delta t) Y_2 ( t + \Delta t )
\]
\begin{equation}
+\lambda_\gamma Y_3 ( t + \Delta t ) = 0.
\label{eq:f3}
\end{equation}
The goal is to solve for the abundances at $t + \Delta t$
such that $f_1 = f_2 = f_3 = 0$.  We may do this by the
Newton-Raphson method.

We begin by considering a Taylor series expansion of the functions
$f_i$:
\begin{equation}
f_i ( {\bf Y} + \delta{\bf Y } ) = f_i ( {\bf Y} )
+ \sum_{k=1}^N \frac{\partial f_i}{\partial Y_j}
\delta Y_j + {\cal O}({\bf \delta Y}^2),
\end{equation}
where $N$ is the total number of equations (species), ${\bf Y}$
is the vector of all $Y$'s, and $\delta Y_j$
represents the change in $Y_j$ over one Newton-Raphson
iteration.  By neglecting terms higher order than linear and
assuming $f_i ( {\bf Y} + \delta{\bf Y} ) = 0$, we
may find
\begin{equation}
\sum_j \alpha_{ij} \delta Y_j = \beta_i  \label{eq:matrix}
\end{equation}
where
\begin{equation}
\alpha_{ij} = \frac{\partial f_i}{\partial Y_j}
\label{eq:Jacobian}
\end{equation}
and
\begin{equation}
\beta_i = -f_i.  \label{eq:rhs}
\end{equation}
The $N$ simultaneous equations in Eq. (\ref{eq:matrix}) make up a
matrix equation $A x = b$.  The matrix $A$ is known as the Jacobian matrix
and has elements $\alpha_{ij}$ given in Eq. (\ref{eq:Jacobian}).  The
right-hand-side vector $b$ has elements $\beta_i$ given in Eq. (\ref{eq:rhs}).
The solution vector $x$ contains the updates $\delta Y_i$
to the abundance.
The matrix equation is solved by
any practical matrix solution method (see below). The corrections
are then added to the solution vector:
\begin{equation}
Y_i^{new} = Y_i^{old} + \delta Y_i
\label{eq:correction}
\end{equation}
for $i = 1$ to $N$ and the process is iterated to convergence, that is, until
the $\delta Y_i$'s are much smaller than the $Y_i$'s and no longer change
the solution.

For our three-element problem, our matrix is
\begin{equation}
A =
 \left(
  \begin{array}{ccc}
    \frac{1}{\Delta t} + N_A \langle \sigma v \rangle \rho Y_2 ( t + \Delta t ) & N_A \langle \sigma v \rangle \rho Y_1 ( t + \Delta t ) - \lambda & -\lambda_\gamma \\
    N_A \langle \sigma v \rangle \rho Y_2 ( t + \Delta t ) & \frac{1}{\Delta t} + N_A \langle \sigma v \rangle \rho Y_1 ( t + \Delta t ) + \lambda & -\lambda_\gamma \\
    -N_A \langle \sigma v \rangle \rho Y_2 ( t + \Delta t ) & N_A \langle \sigma v \rangle \rho Y_1 ( t + \Delta t ) & \frac{1}{\Delta t} + \lambda_\gamma \\
  \end{array}
\right)
\end{equation}
The matrix $A$ changes from one Newton-Raphson iteration to the next
as the values $Y_i$ are improved. Our right-hand side vector
is
\begin{equation}
b = \left(
  \begin{array}{c}
    -f_1 \\
    -f_2 \\
    -f_3 \\
  \end{array}
\right)
\end{equation}
where the values of $f_1,f_2,f_3$ are evaluated from Eqs.
(\ref{eq:f1}-\ref{eq:f3}) using the current values of the $Y$'s.
The solution vector $x$ is given by
\begin{equation}
x = \left(
  \begin{array}{c}
    \delta Y_1 \\
    \delta Y_2 \\
    \delta Y_3 \\
  \end{array}
\right)
\end{equation}
Our matrix equation is thus $Ax = b$.  We iterate until the elements
of $x$ converge to zero.

\section{The Single Zone Example}

The single zone nuclear network example of the libnucnet
distribution uses the above procedure to evolve the abundances. This
example uses an exponential expansion of the temperature and density
such that the density $\rho(t)$ behaves as
\begin{equation}
\rho(t) = \rho_0 \exp({-t/\tau}),  \label{eq:rhot}
\end{equation}
where $\rho_0$ is the initial density and $\tau$ is the expansion
timescale.  The temperature $T_9 = T/10^9$ K is such that $\rho \propto
T_9^3$.  The example reads nuclear, reaction, and initial mass
fraction data from the input XML file along with the initial
density, $T_9$, and $\tau$ from the command line.  The user may also
choose XPath expressions to limit the species or reactions present
in the network.  A $\tau = 0$ indicates a static calculation (i.e.,
constant temperature and density).

The example uses libnucnet API routines to compute the rates at time
$t + \Delta t$, the Jacobian matrix [Eq. (\ref{eq:Jacobian})], and
the right-hand-side vector [Eq. (\ref{eq:rhs})]. It then uses the wn\_matrix
arrow solver or GNU
Scientific Library (gsl) matrix routines to solve the resulting
matrix equations [Eq. (\ref{eq:matrix})].  Newton-Raphson iterations
are performed until the abundances change $Y$ converge.  Convergence is
usually rapid (two or three iterations).  The convergence criteria are
set by changing the parameters D\_MIN and D\_YMIN.  The maximum number of
iterations allowed is set by I\_ITMAX.

Upon convergence of the abundances, the
example then updates the timestep interval $\Delta t$ with the
appropriate API routine and moves on to the next timestep.  The
example prints out the abundances and the abundance changes for each
species with abundance greater than a certain value every $m$
timesteps, where $m$ is input from the user.  The example also
outputs the difference between the sum of the mass fractions and
unity.  Since the sum of all mass fractions should be unity, 1 minus
this sum should always be small.

\section{Multi-Zone Example}

libnucnet easily handles multi-zone networks, as the multi-zone
example in the libnucnet distribution demonstrates.  This example
uses a 1-D set of zones such that each zone mixes with its two
neighbors--the previous zone and the next zone.
The mixing time is the same between all zones.  The
nuclear data, reaction data, and abundance data are read from an
input xml file while the temperature and density of the zones are
read from an input text file.  The mixing time $\tau_{mix}$ between
zones and the duration of the calculation (in seconds) are entered
from the command line.  The user can also enter XPath parameters to
limit the species or reactions included in the calculation.

The rate of change of the abundance of species $i$ in zone $k$ is
given by
\begin{equation}
\frac{dY_i^{(k)}}{dt} = (nuclear\ reaction\ terms) +
\frac{Y_i^{(k-1)}}{\tau_{mix}} - 2 \frac{Y_i^{(k)}}{\tau_{mix}} +
\frac{Y_i^{(k+1)}}{\tau_{mix}},  \label{eq:dydtmix}
\end{equation}
where the $nuclear\ reaction\ terms$ are the normal terms in the
single-zone network.  For the first zone ($k = 1$), the equation
reads
\begin{equation}
\frac{dY_i^{(1)}}{dt} = (nuclear\ reaction\ terms)
-\frac{Y_i^{(1)}}{\tau_{mix}} + \frac{Y_i^{(2)}}{\tau_{mix}}
\label{eq:dydtmix1}
\end{equation}
since there is no previous zone with which to mix.  For the last
zone $(k = M)$, the equation reads
\begin{equation}
\frac{dY_i^{(M)}}{dt} = (nuclear\ reaction\ terms)
-\frac{Y_i^{(M)}}{\tau_{mix}} + \frac{Y_i^{(M-1)}}{\tau_{mix}}
\label{eq:dydtmixM}
\end{equation}
since there is no next zone with which to mix.  When all zones are
incorporated into a single matrix, the result is a block diagonal
matrix with a lower band giving the mixing between the given zone
and the previous one and an upper band giving the mixing between the
given zone and the next one.

The example loops over all the zones setting up the single-zone
Jacobian for each zone and then inserting that into the matrix for
all zones.  Once that is done, the mixing terms are inserted into
the matrix.  The full multi-zone matrix equation is then solved and
the results are then re-apportioned into the separate zones.  The
example uses Newton-Raphson iterations and, upon convergence,
uses the updateTimeStep routine to find the next timestep
interval.  The example prints out the abundances above a threshold
for all zones every timestep.

\section{Matrix Solver}

The single-zone and multi-zone examples provide templates for using
libnucnet in more complicated codes.  Users will probably want to
use their own sparse matrix solvers in place of the routines
used in the examples.  Since the libnucnet routines return the
Jacobian matrix in WnMatrix form, the user should call wn\_matrix
routines to transform the matrix into compressed
sparse row form or Yale sparse matrix form.  See the wn\_matrix
API documentation for a description of those routines.

\end{document}
