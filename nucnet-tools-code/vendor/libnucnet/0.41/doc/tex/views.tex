%///////////////////////////////////////////////////////////////////////////////
% <file type="public">
%
% <license>
%   See the README_module.xml file for this module for copyright and license
%   information.
% </license>
%   <description>
%     <abstract>
%       This is the Webnucleo Report for views libnucnet.
%     </abstract>
%     <keywords>
%       libnucnet Module, webnucleo report, view
%     </keywords>
%   </description>
%
%   <authors>
%     <current>
%       <author userid="mbradle" start_date="2012/03/29" />
%     </current>
%     <previous>
%     </previous>
%   </authors>
%
%   <compatibility>
%     TeX (Web2C 7.4.5) 3.14159 kpathsea version 3.4.5
%   </compatibility>
%
% </file>
%///////////////////////////////////////////////////////////////////////////////
%
% This is a sample LaTeX input file.  (Version of 9 April 1986)
%
% A '%' character causes TeX to ignore all remaining text on the line,
% and is used for comments like this one.

\documentclass{article}    % Specifies the document style.

\usepackage[dvips]{graphicx}
\usepackage{hyperref}

\def\apj{Astrophys. J}

                           % The preamble begins here.
\title{Webnucleo Technical Report: Views in libnucnet} % Declares the document's title.

\author{Bradley S. Meyer}
%\date{December 12, 1984}   % Deleting this command produces today's date.

\begin{document}           % End of preamble and beginning of text.

\maketitle                 % Produces the title.


This technical report describes how to work with views in libnucnet.

\section{Views}

In libnucnet, a Libnucnet\_\_Nuc structure stores data for a collection
of nuclides while
a Libnucnet\_\_Reac structure stores data for a collection of
reactions.  A network, which
is stored as a Libnucnet\_\_Net structure,
is then a combination of the nuclide collection
and the reaction collection.  Valid
reactions in the network are nucleon-number-conserving,
lepton-number-conserving, and charge-conserving reactions in the
Libnucnet\_\_Reac structure among nuclides stored in the network's
Libnucnet\_\_Nuc structure.  A reaction in the network is invalid
if it does not conserve nucleon number, lepton number, or charge, or if
any of the reactants or products are not included among the network's
nuclide collection.

Libnucnet\_\_Nuc and Libnucnet\_\_Reac structures store a considerable
amount of information about their nuclides and reactions.  In many cases
it is desirable to have a subset of these nuclides or reactions.  The
libnucnet API provides the routines Libnucnet\_\_Nuc\_\_extractSubset()
and Libnucnet\_\_Reac\_\_extractSubset(), which, respectively, create
new nuclide and reaction collections from the original collections
through the use of XPath expressions.  Importantly, these new collections
"own" the nuclide or reaction data in the sense that the extractSubset()
routines make copies of those data from the original collections.  Any
modification of the data in the extracted subset does not affect the
data in the original collection.

While these new collections have their uses, it is often preferable to
have a ``view'' of the collection.  A view is a subset of collection
that does not own its own data.  Rather, the data in the view are simply
pointers to the data in the original collection.
An advantage of a view over an extracted subset, then, is that it requires a
considerably smaller amount of memory than the original
collection does.  Also, since the view does not own its data,
modifying the data for a nuclide or reaction in a view modifies the
nuclide's or reaction's data in the original collection.  This means,
for example, that one can modify the data in a nuclide collection
for neon isotopes by getting a view of the collection that only includes
the neon isotopes and iterating over the species in the view and modifying
their data.  This automatically modifies the data for the neon isotopes
in the original collection.

The possible views a user can create are a Libnucnet\_\_NucView, a view
of a nuclide collection, a Libnucnet\_\_ReacView, a view of a reaction
collection, and a Libnucnet\_\_NetView, a view of a network.  It is important
to note that the only reactions included in a Libnucnet\_\_NetView are
ones that are valid for that view.

\section{Libnucnet\_\_NucView}

A Libnucnet\_\_NucView is a view of a nuclide collection.  It is created
with the Libnucnet\_\_NucView\_\_new() routine that takes as arguments
the original Libnucnet\_\_Nuc pointer and an XPath expression to select
the species to include from the original collection in the view.  The
view collection may be accessed via the API routine
Libnucnet\_\_NucView\_\_getNuc(), and the pointer returned from this
routine may be passed into any routine that takes a Libnucnet\_\_Nuc
structure.  The user then frees the view with
Libnucnet\_\_Nuc\_\_free().

For example, to count the number of neon isotopes in an existing nuclide
collection p\_nuc, one could create a view of neon isotopes and get the
number of species in it:
\begin{verbatim}
p_view = Libnucnet__NucView__new( p_nuc, "[a = 10]" );

printf(
  "The number of neon isotopes is %lu.\n",
  Libnucnet__Nuc__getNumberOfSpecies(
    Libnucnet__NucView__getNuc( p_view )
  )
);

Libnucnet__NucView__free( p_view );

\end{verbatim}
It is important to note that p\_nuc and all its data still exist after
these operations since p\_view never owned p\_nuc's data.

\section{Libnucnet\_\_ReacView}

A Libnucnet\_\_ReacView structure is a view of a reaction collection.
It is exactly analogous to a Libnucnet\_\_NucView structure in that it
is created with Libnucnet\_\_ReacView\_\_new(), which takes as arguments
an existing reaction collection and an XPath expression to select the
reactions to include.  The view collection is accessed with
Libnucnet\_\_ReacView\_\_getReac(), and the view is freed with
Libnucnet\_\_ReacView\_\_free().

\section{Libnucnet\_\_NetView}

A Libnucnet\_\_NetView structure is a view of a Libnucnet\_\_Net structure
containing a subset of species of the original structure and valid
reactions among the view's species.  It is created with
Libnucnet\_\_NetView\_\_new(), which takes the original network and
two XPath expressions as arguments.  The first argument is the XPath
expression that selects the nuclides from the original network to include
in the view.  The second XPath expression selects the reactions to include
in the view.  The routine returns a view containing a subset of the
species in the original network and the valid reactions among those species
that satisfy the reaction XPath constraint.

After a view has been created, it is possible to add or remove reactions
from the view with Libnucnet\_\_NetView\_\_addReaction() or
Libnucnet\_\_NetView\_\_removeReaction().  It is worth noting that
since adding a reaction
requires a check that the reaction is valid for the view, this operation
is slower than removing the reaction, which simply deletes the reaction
pointer from the underlying hash.
A network view can be accessed with
Libnucnet\_\_NetView\_\_getNet().  A user can copy a network view
with Libnucnet\_\_NetView\_\_copy(), which returns a new network
view that is a copy of the input one.  The user frees a view with
Libnucnet\_\_NetView\_\_free().

While network views can be used on their own, it is also possible to
store them in Libnucnet\_\_Zones.  This is convenient because the user
can simply lookup a view rather than create it, an operation that requires
numerous checks on reaction validity.  An existing network view can
be added to an existing zone with the command
Libnucnet\_\_Zone\_\_updateNetView(), which adds the view to the
zone if it did not previously exist or replaces the existing view with
the new one.  This routine takes as arguments the zone, three labels for
the view, and the view.  The user subsequently looks up the view from the
zone with the three labels using Libnucnet\_\_Zone\_\_getNet().

It is frequently the case that the logical choices for two of the labels
for a view in a zone are the XPath expressions that created the view,
especially if no reactions have been added to or removed from the view
since it was created.  In this case, the third label can simply be NULL.
The labels, however, need not be XPath expressions.  For example, the
network evolution (change of abundances with time) is computed from an
evolution network view, which has labels (EVOLUTION\_NETWORK, NULL, NULL).
To change the evolution network, then, the user would create a view
and then update the evolution view in p\_zone, the zone of interest.  To
limit the evolution network to $(n,\gamma)$ reactions on nuclei with
$Z \leq 50$, the user would write:
\begin{verbatim}
p_view =
  Libnucnet__NetView__new(
    "[z <= 50]",
    "[reactant = 'n' and product = 'gamma']"
  );

Libnucnet__Zone__updateNetView(
  p_zone,
  EVOLUTION_NETWORK,
  NULL,
  NULL,
  p_view
);

\end{verbatim}
libnucnet routines would then use this network to evolve abundances until
the evolution view was updated again.

Because a network view is created from a parent network,
it is conceivable that the parent network might have changed since the
view was generated.  For example, suppose a user generates
Libnucnet\_\_NetView * p\_view from Libnucnet\_\_Net * p\_net.  Now suppose
the user adds a new species to the nuclide collection 
in p\_net.  p\_view will not include that
species.  At this point the user will want to delete p\_view and generate
a new view.

A user can check whether the parent network of a view has been updated
since the view was generated with the API routine
Libnucnet\_\_NetView\_\_wasNetUpdated().  This routine returns 1 (true)
if the parent network has been updated since the view was generated
or 0 (false) if not.  Checking for an update will allow a user to decide
whether to regenerate a view or not.

A user can iterate over the network views stored in a zone
with Libnucnet\_\_Zone\_\_iterateNetViews() and apply a user defined
Libnucnet\_\_NetView\_\_iterateFunction to them.  To do so, the
user writes a routine with prototype
\begin{verbatim}
int
my_net_view_iterator(
  Libnucnet__NetView * p_view,
  const char * s_label1,
  const char * s_label2,
  const char * s_label3,
  void * p_data
);
\end{verbatim}
In this prototype, p\_data is a pointer to a user-defined
data structure carrying extra data for the routine.
The routine must return 1 (true) for iteration to continue
or 0 (false) for iteration to stop.

The user then iterates over the network views in p\_zone and
applies my\_net\_view\_iterator using p\_data with
\begin{verbatim}
Libnucnet__Zone__iterateNetViews(
  p_zone,
  s_1,
  s_2,
  s_3,
  (Libnucnet__NetView__iterateFunction) my_net_view_iterator,
  p_data
);
\end{verbatim}
This iterates over all network views in p\_zone that have labels
that match s\_1, s\_2, and s\_3 and applies my\_net\_view\_iterator
to each view.  If s\_1, s\_2, or s\_3 is NULL, any view label is a match;
thus, supplying NULL for s\_1, s\_2, and s\_3 will iterate over all
network views in p\_zone.

Once a network view is added to a zone with
Libnucnet\_\_Zone\_\_updateNetView(), the zone owns the view.  This
means that the memory for the view will be freed when the zone is freed.
If a network view has not been added to a zone, it is the user's
responsibility to free the memory with Libnucnet\_\_NetView\_\_free().

As of version 0.19, it is possible to copy all network views from one zone to
another and to clear all network views from a zone without deleting the zone.
Copying network views from one zone to another is done with the
libnucnet API routine Libnucnet\_\_Zone\_\_copy\_net\_views().  This routine
allows a user to avoid having
to create a new network view in the destination
zone, a time consuming process since
all reactions must be checked for validity when the new view is created.
Of course the underlying network for the source zone and the destination zone
must be the same.  libnucnet will invoke error handling if the underlying
networks are not the same.

Clearing all network views from a zone
is done with the libnucnet API routine
Libnucnet\_\_Zone\_\_clearNetViews().  A user might wish to clear the
network views from a zone, for example,
when separate zones contain the results of
separate nucleosynthesis calculations.  If the calculation for a given
zone is completed, the network views in that zone may no longer be needed.
If the user wants to retain the zone, the memory for the views will remain
allocated unless the user clears the network view from the zone.
Doing so will free up memory for the nucleosynthesis
calculations in subsequent zones.

\end{document}
